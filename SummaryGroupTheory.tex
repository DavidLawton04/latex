\documentclass{article}
\usepackage{graphicx}
\usepackage{mathtools}
\usepackage{xfrac}
\usepackage{amsmath, amssymb}
\usepackage{listings}
\usepackage{float}
\usepackage{wrapfig}
\usepackage{tikz}
\usepackage{fullpage}
\usepackage{hyperref}
\usepackage{mathalpha}
\usepackage{tikz}
\usepackage{cite}
\usepackage{amsthm}

\newtheorem{theorem}{Proposition}[section]
\newtheorem{corollary}{Corollary}[theorem]
\newtheorem{lemma}[theorem]{Lemma}

\theoremstyle{definition}
\newtheorem{definition}{Definition}[section]

\theoremstyle{remark}
\newtheorem*{remark}{Remark}
\newtheorem*{example}{Example}
\newtheorem*{notation}{Notation}

\title{Introduction to Group Theory: Summary}
\author{Based on the lectures of Florian Naef\\David Lawton\\22337087}
\date{20th May 2024.}

\begin{document}

\maketitle
\vfill
\tableofcontents

\newpage

\section{Lecture 1: Definitions \& Examples}
\begin{definition}
	A \textbf{group} is pair $(G,m)$ such that $G$ is a \textbf{set} and $m: G\times G \rightarrow G$ is a mapping from $G$ to itself s.t.
\end{definition}
\begin{itemize}
	\item $G$ is associative under $m$, ie. $m(a,(b,c))=m((a,b),c) ~\forall ~a,b,c\in G$.
	\item $G$ has a unit, ie. $\exists ~e\in G$ s.t. $ m(e,g)=m(g,e)=g~\forall ~ g\in G$.
	\item Each element of $G$ has an inverse, ie. $\forall~a\in G, \exists ~b\in G$ s.t. $ m(a,b)=m(b,a)=e$.
\end{itemize}
\hrule
\vspace{3mm}
\begin{remark}
	We usually write $m(a,b)$ as $a*b$, $ a\cdot b$, or $ab$. Associativity becomes $a(bc)=(ab)c$. \\ We also write the inverse of element $a$ as $a^{-1}$. \\ The notation $(G,m) $ is rewritten simply as $G$ for convenience.\\
\end{remark}


\begin{example}
	~\\
	\begin{enumerate}

		\item \vspace{-2mm}$G=\lbrace e\rbrace$ (Trivial group)
		\item $(\mathbb{Z}, +)$, $e=0$, $a^{-1}=-a$ (Integers under addition)
		\item $(\mathbb{Q}, +)$ (Rational numbers under addition)\footnote{For any field $F$, $(F,+)$ and $(F\setminus\lbrace 0\rbrace, *)$ are groups.}
		\item $(\mathbb{Q}^x=\mathbb{Q}\setminus \lbrace 0\rbrace, *)$, $e=1$, $a^{-1}=\frac{1}{a}$
		\item $GL(n,\mathbb{R})= \lbrace n\times n \text{ matrix } A \text{ with entries in } \mathbb{R}|\det A \neq 0 \rbrace$, $e = \mathbb{I}$, $A^{-1}=A^{-1}$ (General linear group)
		\item $S(X)=\lbrace f:X\times X \rightarrow X | f \text{ bijective}\rbrace$ , $ e=Id_X$, $f^{-1}=f^{-1}$
	\end{enumerate}
\end{example}
\hrule
\vspace{3mm}~\\
\begin{definition}
	A group is \textbf{abelian} if all elements of the set are commutative under the mapping, ie. for group $G=(G,m)$, $ab=ba~\forall~a,b\in G$.\footnote{1, 2, 3 abelian. 4, 5 generally non-abelian ($n\geq 2$ in 4, $|X| \geq 3$ in 5.)} Note: $a*b$ often written as $a+b$ for abelian groups.\\
\end{definition}
\begin{theorem} For any group the following is true.
	\begin{enumerate}
		\item The unit is unique
		\item For each $a\in G$, $a^{-1}$ is uniquely determined.
		\item $(a^{-1})^{-1}=a$
		\item $(ab)^{-1}=b^{-1}a^{-1}$
		\item For any $a_1, ..., a_n$, the value of $a_1\cdot ... \cdot a_n$ is independent of bracketing.
	\end{enumerate}
\end{theorem}
\begin{proof} Each numbered proof correspond to the respective number in the proposition.
	\begin{enumerate}
		\item Suppose $e, e\prime$ are both units of group $G$. Then $e=e' e = e'$. $\square$
		\item Given $a \in G$, suppose $\exists~ b_1,b_2 \in G$ s.t. they both satisfy the conditions of the inverse of $a$. Then $b_1=b_1 e = b_1(ab_2)=(b_1a)b_2=eb_2=b_2$. $\square$
		\item Let $b=(a^{-1})^{-1}$, therefore $ba^{-1}=e=a^{-1} b$. $a$ satisfies this, and since the inverse is uniquely determined, $a=b=(a^{-1})^{-1}$.
		\item $ ab(b^{-1}a^{-1})=a(bb^{-1})a^{-1}=aa^{-1}=e$. Similar for $(b^{-1}a^{-1})ab$. Therefore $b^{-1}a^{-1} $ satisfies the conditions of the inverse of $ab$, and is therefore equal to $(ab)^{-1}$ since the inverse is uniquely determined.
		\item Proof by induction. Let $f(a_1,...,a_n)$ be a bracketing of $a_1,...,a_n$. Define $f(a_1,...,a_n)=(a_1(...(a_{n-1}a_n)...)):=m_n(a_1,...,a_n)$.
		      \vspace{2mm}\\
		      Induction on $n$:\\ $n=1,2$: $m(a_1)=a_1,~~m_2(a_1,a_2)=m(a_1,a_2)$.\\$n\geq 3$: $f=m(f_1(a_1,...,a_k),f_2(a_{k+1},...,a_n))$.\\
			      By ind. hyp. $f_1=m_k$, $f_2=m_{n-k}$.\\
			      It remains to show that $m(m_k,m_{n-k})=m_n~\forall~ k$.

		      \begin{align*}
			      k=1:   & m(a_1,m_{n-1}(a_2,...,a_n))=m_n(a_1,...,a_n).                                                         \\
			      k > 1: & m(m_k(a_1,...,a_k),m_{n-k}(a_{k+1},...,a_n))=m(m(a_1,m_{k-1}(a_2,...,a_k)),m_{n-k}(a_{k+1},...,a_n)). \\ =&m(a_1,m(m_{k-1}(a_2,...,a_k),m_{n-k}(a_{k+1},...,a_n))) \text{ by associativity.}\\
			      =      & m(a_1,m_{n-1}(a_2,...,a_n))=m_n(a_1,...,a_n)
		      \end{align*}
	\end{enumerate}
\end{proof}
\hrule

\begin{remark}
	\vspace{1mm}Either of left or right inverse, uniquely characterise $a^{-1}$.\\
\end{remark}
\hrule
\vspace{3mm}~\\
\begin{theorem}
	Left and right cancellation hold in any group.
	\begin{align}
		ax & =ay\therefore x=y \\
		xa & =ya\therefore x=y
	\end{align}
\end{theorem}
\begin{proof}
	Multiply by $a^{-1}$ from left, right respectively.\\
\end{proof}
\hrule
\begin{remark}
	\vspace{1mm}Let  $(G, m),~m:G\times G \rightarrow G$ satisfy:\\
	\begin{itemize}
		\item $m(a, m(b,c))=m(m(a,b),c)$ (Associativity)
		\item $\exists~e\in G$ s.t. $m(e,g)=g,~\forall~g\in G$. (Left-unit)
		\item $\forall ~ a \in G,~\exists~b\in G$ s.t. $m(b,a)=e$. (Left-inverse)
	\end{itemize}
	then $(G,m)$ is a group.\\
\end{remark}
\begin{notation}
	\begin{align}
		x^n=      & x\cdot (x, ~ n-2 \text{ times}) \cdot x,~x^{-n}=x^{-1}\cdot (x^{-1}, ~ n-2 \text{ times}) \cdot x^{-1} \\
		n\cdot x= & x+x+x+...+x,~-n\cdot x = (-x)+(-x)+(-x)+...+(-x) \text{ (For abelian)}
	\end{align}
\end{notation}
\hrule
\begin{definition}
	The \textbf{order} of $x\in G$ is the smallest $n\in \mathbb{Z}^+$ s.t. $x^n=e$. The order is denoted $|x|= n$.\\
\end{definition}
\hrule
\begin{example}
	\begin{itemize}
		\item $G=\mathbb{C}^{\times}, x=i, |x|=4.$
		\item $G=\mathrm{GL}(2,\mathbb{R}), x=\begin{pmatrix}0&-1\\1&1\\\end{pmatrix}, |x|=6$
	\end{itemize}
\end{example}
\newpage
\section{Lecture 2: Integers Modulo n and the Quaternion Group}
\subsection{Integers Modulo n: \texorpdfstring{$\mathbb{Z}/n\mathbb{Z}$}{}}
\begin{definition}
	Let $a,b\in \mathbb{Z}$. We say $a,b$ have the same residue mod $n$, and write $a\equiv b~(\text{mod n})$ if $\exists~k\in \mathbb{Z}$ s.t. $a-b=k\cdot n$.
\end{definition}
Given $a\in \mathbb{Z}$ denote by
\begin{align*}
	\bar{a} & =\lbrace b\in \mathbb{Z}|b\equiv a~ (\text{mod n})\rbrace                   \\
	        & =\lbrace a + kn \in \mathbb{Z}| k\in \mathbb{Z}\rbrace \subseteq \mathbb{Z}
\end{align*}
and define
\begin{equation}
	\mathbb{Z}/n\mathbb{Z}=\lbrace \bar{a} \subseteq \mathbb{Z}|a\in \mathbb{Z}\rbrace
\end{equation}

\begin{lemma}
	\begin{itemize}
		\item $a\equiv b ~(\text{mod n}) ~\Leftrightarrow ~\bar{a}=\bar{b}$
		\item $\mathbb{Z}/n\mathbb{Z} = \lbrace\bar{0}, \bar{1}, \bar{2}, ..., \overline{n-1}\rbrace $
	\end{itemize}
\end{lemma}
\begin{proof}
	\begin{align*}
		a\equiv b~(\text{mod n})~ & \Rightarrow ~ a= b + k\cdot n                                    \\
		                          & \Rightarrow ~ b = a - k\cdot n = a + l\cdot n ,~ l\in \mathbb{Z} \\
		                          & \Rightarrow ~ b=a~(\text{mod n})
	\end{align*}
	\begin{align*}
		\mathbb{Z}/n\mathbb{Z}=\lbrace \bar{a}\subseteq \mathbb{Z}|a\in \mathbb{Z}\rbrace,~ \bar{a} & =\lbrace a + kn \in \mathbb{Z}| k\in \mathbb{Z}\rbrace  \\
		\Rightarrow ~\forall~a<n, ~ \bar{a}                                                         & =\lbrace a + kn \in \mathbb{Z}| k\in \mathbb{Z}\rbrace, \\ \forall ~ a\geq n,~\bar{a}&=\lbrace n+b +kn\in \mathbb{Z}|k\in\mathbb{Z}, a=n+b\rbrace\\
		                                                                                            & =\lbrace b+(k+1)n|k\in\mathbb{Z}\rbrace=\bar{b}
	\end{align*}
	\begin{align*}
		\Rightarrow ~ \forall ~ a\geq n, ~ \bar{a} & =\overline{a-n}                                                                                                         \\
		\Rightarrow  ~ \mathbb{Z}/n\mathbb{Z}      & =\lbrace \bar{a}\subseteq \mathbb{Z}|a\in \mathbb{Z}, ~ a<n\rbrace=\lbrace \bar{0}, \bar{1}, ..., \overline{n-1}\rbrace
	\end{align*}
\end{proof}
\begin{theorem}
	The assignment $m(\bar{a},\bar{b})=\overline{a+b}$ is well defined, and $(\mathbb{Z}/n\mathbb{Z},m) $ is an abelian group.\\
\end{theorem}
\begin{proof}
	Let $\bar{a_1}=\bar{a_2},~\bar{b_1}=\bar{b_2}$, this implies that $a_1\equiv a_2 $ (mod n), $b_1\equiv b_2$ (mod n). It is then necessary that $a_1+b_1=a_2+b_2$ (mod n). Thus $\overline{a_1+b_1}=\overline{a_2+b_2}$, and so $m$ is well defined.\\
	\indent It remains to show that $(\mathbb{Z}/n\mathbb{Z},m)$ is a group. Therefore we must show that it is associative, and contains a left-unit, and left-inverse.\\
	\indent Let $G=(\mathbb{Z}/n\mathbb{Z},m)$. For $a,b,c\in G$,
	\begin{align*}
		m(a,b)=a+b, ~m(b,c)=     & b+c                \\
		\Rightarrow m(a,m(b,c))= & a+b+c=m(m(a,b),c),
	\end{align*}
	therefore $G$ is associative. \\ \indent To show the existence of the left-unit, we must show there exists $e\in G$ s.t. $e+g=g~\forall~g\in G$. For $m(\bar{a},\bar{b})=\overline{a+b}=\bar{b}$, where $\bar{a},\bar{b}\in G$, it is clear from this that $a=k\cdot n$ for some $k\in \mathbb{Z}$. Therefore $a=0$ (mod n). This implies that $\bar{a}=\bar{0}$. Therefore $\bar{0}$ is the left-unit of $G$ (and consequently right-unit as abelian).\\
	\indent To show the existence of the left-inverse, we must show that $\forall~a\in G, ~\exists ~ b\in G$ s.t. $m(a,b)=e$. If $m(\bar{a}',\bar{b}')=\bar{0}$, then $a'=-b'$ (mod n). Therefore, $a'=n-b'$ (mod n), which implies that $\bar{a}=\overline{n-b}'$. Hence, $\forall~\bar{g}\in G, ~\bar{g}^{-1}=\overline{n-g}$.\\
	\indent $G=(\mathbb{Z}/n\mathbb{Z},m) $ then satisfies all the required conditions of a  group. To show that $G$ is abelian, one must only note that $a+b=b+a~\forall ~ a,b\in \mathbb{Z}$ which implies that $\overline{a+b}=\overline{b+a}$, and consequently $m(\bar{a},\bar{b})=m(\bar{b},\bar{a})~\forall \bar{a},\bar{b}\in G$.\\
\end{proof}

\hrule
\vspace{2mm}~\\
\begin{notation}
	We write $a=\bar{a}$, for example, in $\mathbb{Z}/5\mathbb{Z}$, we write $2+3=0$.\\
\end{notation}

\hrule
\vspace{2mm}~\\
\begin{lemma}
	$1\in\mathbb{Z}/n\mathbb{Z}$ has order $n$.
\end{lemma}
\begin{proof}
	\begin{align*}
		n\cdot 1 & = n = 0                      \\
		k\cdot 1 & =k\neq 0 \text{, for } 0<k<n
	\end{align*}
\end{proof}

\subsection{Quaternion Group}
Let $Q_8=\lbrace \pm 1,\pm i,\pm j,\pm k \rbrace$, with $m:Q_8\times Q_8\rightarrow Q_8$ given by:
\begin{align*}
	i^2 & =j^2=k^2=-1       \\
	ij  & =k,~ki=j,~jk=i    \\
	ji  & =-k,~ik=-j,~kj=-i
\end{align*}
where signs manipulate as expected.\\

\begin{theorem}
	$(Q_8,m)$ is a group.
\end{theorem}
\begin{proof}
	Simple to show associativity, left-unit, left-inverse. Not done here.\\
\end{proof}

\newpage
\section{Lecture 3: Generators-Relations}
Given a set $r_1,r_2,r_3,...,r_l$ of words (relations) in $g_1^{\pm},g_2^{\pm},...,g_k^{\pm}$ (generators). We can define a group
\begin{equation}
	G=\langle g_1,...,g_k|r_1,...,r_l\rangle
\end{equation}
This is called the presentation of a group. We will define the group more precisely later.\\
\indent Elements of $G$ are words (combinations) of $g_1^{\pm},g_2^{\pm},...,g_l^{\pm}$ under the equivalence relation given by
\begin{itemize}
	\item removing/adding $g_ig_i^{-1}, g_i^{-1}g_i,e$,
	\item replacing an occurence of $r_i$ with $e$.
\end{itemize}
\vspace{2mm}~\\
\hrule
\vspace{2mm}~\\
\begin{example} Dihedral Group
	\begin{equation}
		D_{2n}=\langle r,s|r^n=s^2=(sr)^2=e\rangle
	\end{equation}
	Let $w$ try to enumerate all the elements of $D_{2n}$:
	If $f$ is any word in $r^{\pm},s^{\pm}$, use $r^-1=r^{n-1}$ and $s^{-1}=s$ to get a word in $r,s$. Since $s^2=e$, we can assume
	\begin{equation}
		f=r^{i_1}sr^{i_2}s...sr^{i_l},~i_j>0
	\end{equation}
	and then use $sr=(sr)^{-1}=r^{-1}s^{-1}=r^{n-1}$ to move the terms around and reach the form $f=sr^i$ or $f=r^i$.\\
	\begin{equation}
		\Rightarrow D_{2n}=\lbrace e,r,...,r^{n-1},s,rs,...,r^{n-1}s\rbrace
	\end{equation}
	These elements are not necessarily distinct.\\
	\indent $D_{2n}$ is the group of symmetries on a regular n-gon. $D_{2n}$ can be realised as
	\begin{equation}
		r=
		\begin{pmatrix}
			\cos(\theta) & -\sin(\theta) \\
			\sin(\theta) & \cos(\theta)
		\end{pmatrix}
		,~\theta = \frac{2\pi}{n},~s=
		\begin{pmatrix}
			-1 & 0  \\
			0  & -1
		\end{pmatrix}
	\end{equation}
\end{example}
\begin{remark}~
	\begin{itemize}
		\item $G=\langle \text{gen.}|\text{rel.}\rangle$ is always a group.
		\item Generally, it is difficult to decide for $x\in G$, if $x=e$.
	\end{itemize}
\end{remark}
\newpage
\section{Lecture 4: Symmetric Group}
The symmetric group is the group of bijective maps from a set of n elements to the its These map between permutations of these n elements.
\begin{equation}
	\label{eq: SymmetricGroup}
	S_n=\lbrace \sigma:\lbrace 1,2,...,n\rbrace\rightarrow\lbrace1,2,...,n\rbrace|\sigma \text{ bij}.\rbrace
\end{equation}
Each element of $S_n$ can be written in the form of the permutation it maps the original set to, ie. $\left(\sigma(1),...,\sigma(n)\right)$. \\
\begin{equation}
	\sigma = \begin{pmatrix}
		2 & 1 & 3
	\end{pmatrix} \in S_3
	\text{ or } \sigma =\begin{pmatrix}
		1 & 2 & 3 \\ 2&1&3
	\end{pmatrix}\in S_3.
\end{equation}
as well as this these maps can be decomposed into cycles. For example,
\begin{align*}
	\sigma=\begin{pmatrix}
		       1 & 2 & 3 & 45 \\ 3&4&5&2&1
	       \end{pmatrix}\in S_5                                        \\
	1\rightarrow 3 \rightarrow 5 \rightarrow 1,~2\rightarrow 4 \rightarrow 2 \\
	\Rightarrow \sigma = \begin{pmatrix}
		                     1 & 3 & 5
	                     \end{pmatrix}\begin{pmatrix}
		                                  2 & 4
	                                  \end{pmatrix}
\end{align*}
this is called cycle decomposition.\\
\vspace{2mm}~\\
\begin{definition}
	Given $a_1,a_2,...,a_l\in \lbrace 1,...,n\rbrace$, all distinct, we define an \textbf{`l-cycle'}: $S_n\ni\sigma:=\begin{pmatrix}
			a_1 & ... & a_l
		\end{pmatrix}$\footnote{$\begin{pmatrix}
				a_1 & a_2 & ... & a_l
			\end{pmatrix}=\begin{pmatrix}
				a_2 & ... & a_l & a_1
			\end{pmatrix}$}
	by the formula
	\begin{equation}
		\sigma(x)=
		\begin{cases}
			a_{j+1} & \text{, if }x=a_j \\
			x       & \text{, else}
		\end{cases}
	\end{equation}
\end{definition}
\begin{lemma}
	Let $\sigma=\begin{pmatrix}
			a_1 & a_2 & ... & a_l
		\end{pmatrix}$
	and $\tau = \begin{pmatrix}
			b_1 & b_2 & ... & b_k
		\end{pmatrix}$ be such that $\lbrace a_1,a_2,...,a_l\rbrace \wedge \lbrace b_1,b_2,...,b_k\rbrace = \emptyset$. Then $\sigma\cdot \tau=\tau\cdot\sigma$.\\
\end{lemma}
\begin{proof}
	Since the two sets, $A=\lbrace a_1,a_2,...,a_l\rbrace,B=\lbrace b_1,b_2,...,b_k\rbrace$ are disjoint, $\sigma(b_i)=b_i,\tau(a_i)=a_i$ and $\sigma(x)=\tau(x)=x~\forall~x$ not in $A,B$. Therefore it follows that $\sigma\cdot\tau(b_i)=\tau\cdot\sigma(b_i)=b_{i+1}$, $\sigma\cdot\tau(a_i)=\tau\cdot\sigma(a_i)=a_{i+1}$, and $\sigma\cdot\tau(x)=\tau\cdot\sigma(x)=x~\forall~x$ not in $A,B$. Therefore $\forall ~ g\in \lbrace 1,...,n\rbrace,\tau\cdot\sigma=\sigma\cdot\tau$.\\
\end{proof}

\begin{theorem}
	Every $\sigma \in S_n$ admits a decomposition into disjoint cycles.
\end{theorem}
\begin{proof}
	For some $\sigma\in S_n$, and $i\in\lbrace1,...,n\rbrace$ be s.t. $i=\mathrm{min}\lbrace j\in\lbrace1,...,n\rbrace|\sigma(j)\neq j\rbrace$. For some $l_1,l_2$, we have $\sigma^{l_1}(i)=\sigma^{l_2}(i)$. Therefore, $\sigma^{l_1-l_2}(i)=i$, and w.l.o.g. $l=l_1-l_2>0$. Set $\sigma=\begin{pmatrix}
			i & f(i) & ... & f^{l-1}(i)
		\end{pmatrix}$.\\
	The proof is continued by replacing $\sigma$ with $\sigma_1^{-1}\cdot\sigma$, and $i=\mathrm{min}\lbrace j \in\lbrace1,...,n\rbrace| \sigma(j)\neq j\rbrace$ with $i=\mathrm{min}\lbrace j\in \lbrace1,...,n\rbrace\mathrm{min}\lbrace j \in\lbrace1,...,n\rbrace|\sigma_1^{-1}\cdot\sigma(j)\neq j\rbrace$ and repeating.\\
\end{proof}
\hrule
\vspace{2mm}
\begin{remark}
	Not every product of cycles is a cycles decomposition. ie. If the cycles are not disjoint.\\
\end{remark}
\hrule
\newpage
\section{Lecture 5: The Category of Groups.}
Consider $G=\lbrace e,a,b,c\rbrace$ with multiplication.
\begin{equation}
	\begin{tabular}{c|cccc}
		~ & e & a & b & c \\
		\hline
		e & e & a & b & c \\
		a & a & e & c & b \\
		b & b & c & a & e \\
		c & c & b & e & a \\
	\end{tabular}
\end{equation}
Note that if we assign $a=2,b=1,c=3$ this 'is' (isomorphic to, this will be defined later) $\mathbb{Z}/4\mathbb{Z}$.\\
\begin{definition}
	\label{GroupHomomorphism}
	Let $G,H$ be groups. A group \textbf{homomorphism} is a map $\varphi:G\rightarrow H$ s.t.
	\begin{equation}
		\varphi(a\cdot_Gb)=\varphi (a)\cdot_H\varphi(b)
	\end{equation}
	($\cdot_G,\cdot_H$ denote the binary maps of $G,H$ respectively.)\\
\end{definition}
\begin{definition}
	If group homomorphism $\varphi$ is a bijection, then it is called a \textbf{group isomorphism}. In this case $G,H$ are \textbf{isomorphic}.\\
\end{definition}
\begin{theorem}
	Let $\varphi:G\rightarrow H$ be a group homomorphism. Then
	\begin{enumerate}
		\item $\varphi(e_G)=e_H$
		\item $\varphi(a^{-1})=\varphi(a)^{-1}$
	\end{enumerate}
\end{theorem}
\begin{proof}
	From the definition of a group homomorphism:
	\begin{align*}
		\varphi(e_G\cdot_Gb) & =\varphi(e_G)\cdot_H\varphi(b) \\
		                     & =\varphi(b)
	\end{align*}
	This implies
	\begin{align*}
		\varphi(b)=\varphi(e_G)\cdot_H\varphi(b)
	\end{align*}
	which can only be true if
	\begin{equation}
		\varphi(e_G)=e_H
	\end{equation}
	proving the first condition. The second condition begins in a similar way, from the definition of a group homomorphism we know
	\begin{align*}
		\varphi(a\cdot_G a^{-1})=\varphi(a)\cdot_H\varphi(a^{-1})
	\end{align*}
	but
	\begin{align*}
		\varphi(a\cdot_Ga^{-1})=\varphi(e_G)=e_H         \\
		\Rightarrow \varphi(a)\cdot_H\varphi(a^{-1})=e_H \\
		\Rightarrow \varphi(a^{-1})=\varphi(a)^{-1}
	\end{align*}
\end{proof}

\begin{definition}
	$|G|$ is called the \textbf{order} of $G$. This is defined as the number of elements in the group for a finite group (group in which the underlying set is finite) or infinity for a non-finite group.\\
\end{definition}
\begin{theorem}
	Let $G$ be a group of order $2$. Then $G\cong \mathbb{Z}/2\mathbb{Z}$ (Isomorphic)
\end{theorem}
\begin{proof}
	$G$ group $\Rightarrow$ $\exists~e\in G$, and $ a\in G$ s.t $a\neq e$. Define map $\varphi:\mathbb{Z}/n\mathbb{Z}\rightarrow G$.
	\begin{align*}
		0\mapsto & e \\
		1\mapsto & a
	\end{align*}
	We must then check that $\varphi(x+y)=\varphi(x)\varphi(y)~\forall~,y\in \mathbb{z}/n\mathbb{Z}$:\\
	\begin{center}
		\begin{tabular}{c|c|c}
			$x$ & $y$ & ~                                                                     \\
			\hline
			$0$ & $0$ & Trivially true                                                        \\
			\hline
			$1$ & $1$ & $\varphi(1+1)=\varphi(0)=e$                                           \\
			~   & ~   & $\varphi (1)\varphi(1)=a^2$                                           \\
			~   & ~   & Only true if $a^2=e$, which must be true since $a^2=a$ implies $a=e$. \\
			\hline
			$0$ & $1$ & True since $\varphi(1)=a$, $e\varphi(1)=a$                            \\
			\hline
			$1$ & $0$ & True since $\varphi(1)=a$, $e\varphi(1)=a$                            \\
		\end{tabular}
	\end{center}
\end{proof}
\begin{theorem}~
	\begin{enumerate}
		\item Let $f:H\rightarrow K$, $g:G\rightarrow H$ be group homomorphisms, then so is $f\circ g$.
		\item Let $f:H\rightarrow K$ be a group isomorphism, then so is $f^{-1}$.
	\end{enumerate}
\end{theorem}
\begin{proof}~
	\begin{enumerate}
		\item $ (f\circ g)(ab)=f(g(ab)=f(g(a)g(b))=f(g(a))f(g(b))=(f\circ g)(a)(f\circ g)(b)$\\
		      Therefore $f\circ g$ satisfies the condition of a group homomorphism.
		\item $f(f^{-1}(ab))=ab=f(f^{-1}(a))f(f^{-1}(b))$, then since $f$ is injective, $f(p)=f(q)$ implies $p=q$. Therefore $f^{-1}(ab)=f^{-1}(a)f^{-1}(b)$, and $f^{-1}$ is then a group homomorphism.
	\end{enumerate}
\end{proof}
\newpage
\section{Lecture 6: Group Actions}
\begin{definition}
	A \textbf{group action} of a group $G$ on a set $X$ is a map $G\times X\rightarrow X$ written as $(g,x)\mapsto g.x$ s.t.
	\begin{itemize}
		\item $g_1\cdot (g_2\cdot x)=(g_1g_2)\cdot x$
		\item $e\cdot x = x$
	\end{itemize}
	We sometimes write $G\circlearrowright X$.\\
\end{definition}
A map of sets $G\times X\rightarrow X$ can be equivalently given by
\begin{align*}
	G & \rightarrow^{\rho}\lbrace f:X\rightarrow X\rbrace \\
	g & \mapsto\rho(g)
\end{align*}
where $\rho(g)(x)=g\cdot x$.\\
\begin{theorem}
	A map $G\times X\rightarrow X$ defines a group action iff\footnote{If and only if} the corresponding map $G\rightarrow^{\rho}\lbrace f:X\rightarrow X\rbrace$ is s.t. $\rho(g)\in S_X ~\forall~g\in G$ and $\rho:G\rightarrow S_X$ is a group homomorphism.\footnote{$S_X$ is the symmetric group on set $X$}
\end{theorem}
\begin{proof}
	\begin{align*}
		\rho (g_1)(\rho(g_2)(x))=\rho(g_1g_2)(x)~ & \Leftrightarrow ~\rho(g_1)\circ\rho(g_2)=\rho(g_1g_2) \\&\Leftrightarrow ~\rho(e)=\mathrm{id}_X
	\end{align*}

	\begin{align*}
		``\Rightarrow":~ & \rho(g)\circ\rho(g^{-1})=\rho(gg^{-1})=\rho(e)=\mathrm{id}_X ~\Rightarrow\rho(g)\text{ surj.}\footnotemark \\ &\rho(g^{-1})\circ\rho(g)=\rho(g^{-1}g)=\mathrm{id}_X~\Rightarrow\rho(g)\text{ inj.}\footnotemark
	\end{align*}\footnotetext{ Maps are surjective if and only if a right inverse exists for each element. Maps are injective if and only if a left inverse exists for each element.}
	This implies that $\rho(g)\in S_X$, as the requirement is that the map is bijective. Since the first line, at the start of the proof, is true for any for any group action, this implies that $\rho:G\rightarrow S_X$ is a group homomorphism. Then, since all steps taken are reversible, the same process can be taken in reverse to show the bijectivity of $\rho(g)$, and $\rho$ being a group homomorphism, implies that the map $G\times X\rightarrow X$ is a group action, proving $``\Leftarrow"$. \\
\end{proof}
\hrule
\vspace{2mm}
\begin{example}~
	\begin{enumerate}
		\item Trivial action: For any $X$, we define $G\times X\rightarrow X$ s.t. $(g,x)\mapsto x$.
		\item Defining action of $S_X$ on $X$: $S_X\times X\rightarrow X$, $(\sigma,x)\mapsto\sigma(x)$.\\ We can claim that the map $\rho$ from above is in this scenario $\mathrm{id}:S_X\rightarrow S_X$.
		\item $G$ acting on itself by \\
		      \begin{tabular}{ccc}
			      $\rho_l : G\times G\rightarrow G$    & $(g,x)\mapsto gx$       & Called \textbf{left regular action}.  \\
			      $\rho_r:G\times G\rightarrow G$      & $(g,x)\mapsto xg^{-1}$  & Called \textbf{right regular action}. \\
			      $\rho_{adj.}:G\times G\rightarrow G$ & $(g,x)\mapsto gxg^{-1}$ & Called \textbf{adjoint action}.
		      \end{tabular}
		      Verification of each as a group action:
		      \begin{align*}
			      \rho_l:      &                                                                                                                                    \\
			                   & \text{ We can let the map corresponding to }\rho_l,\phi:G\rightarrow\lbrace f:X\rightarrow X\rbrace \text{ be s.t. }\phi(g)(x)=gx. \\
			                   & \text{ Then } \phi(g_1)(\phi(g_2)(x))=\phi(g_1)(g_2x)=g_1g_2x=\phi(g_1g_2)(x).                                                     \\
			                   & \Rightarrow ~\phi(g_1)\circ\phi(g_2)=\phi(g_1g_2), \phi(e)=e                                                                       \\
			                   & \text{ As in the above proof, this implies both left and right inverses therefore }\rho_l\in S_X                                   \\
			      \rho_r:      &                                                                                                                                    \\
			                   & \text{ Analagous to }\rho_l                                                                                                        \\
			      \rho_{adj.}: &                                                                                                                                    \\
			                   & \text{ Similar to }\rho_l, \text{ define } \phi(g)(x)=gxg^{-1}. \text{ Then } \phi(g_1)(\phi(g_2)(x))=\phi(g_1)(g_2xg_2^{-1})      \\&=g_1g_2xg_2^{-1}g_1^{-1}=(g_1g_2)x(g_1g_2)^{-1}=\phi(g_1g_2)(x).\\
			                   & \Rightarrow ~\phi(g_1)\circ\phi(g_2)=\phi(g_1g_2), \phi(e)=e                                                                       \\
			                   & \text{ Same conclusion can be drawn as in the case of } \rho_l.
		      \end{align*}
		\item $D_{2n}\times\lbrace1,...,n\rbrace\rightarrow\lbrace1,...,n\rbrace:$
		      \begin{align*}
			      (r^i,j)  & \mapsto i+j \text{ mod }n \\
			      (r^is,j) & \mapsto i-j\text{ mod }n
		      \end{align*}
		      \begin{figure}[H]
			      \begin{center}
				      \includegraphics[width=0.2\textwidth]{/home/dj-lawton/Documents/Tikz/tikzit_image4.png}
				      \caption{\label{fig: Dihedral}Illustration of the action of the $D_{12}$ group on the set $\lbrace 1,...,n\rbrace$. The element $s$ of $D_{2n}$ reflects across the vertical, and the element $r$ rotates by $\frac{2\pi}{n}$ in the anti-clockwise direction. }
			      \end{center}
		      \end{figure}
	\end{enumerate}
\end{example}
\newpage
\section{Lecture 7: Subgroups}
\begin{definition}
	A subset $H\leq G$ of a group $(G,m)$ is a \textbf{subgroup} if the restriction of $m$ to $H\times H$ turns $H$ into a group. We write $H\leq  G$ in that case.\\
\end{definition}
\hrule
\vspace{2mm}
\begin{remark}
	To specify, $H$ is required to have contain the unit, inverses and be associative, it is also required to have closure under $m$. That is to say $\forall ~a,b\in H,m(a,b)\in H$.\\
\end{remark}
\hrule
\begin{theorem}
	$H\subseteq G$ is a subgroup iff
	\begin{itemize}
		\item $H$ is non-empty.
		\item if $a,b\in H$, $ab^{-1}\in H$.
	\end{itemize}
\end{theorem}
\begin{proof}
	\begin{align*}
		"\Rightarrow": & ~H\leq G\Rightarrow ~\exists~ e\in H\text{ s.t. } ea=a~\forall~a\in H.                                                                   \\
		               & \therefore H\text{ non-empty.}                                                                                                           \\
		               & ~H\leq G\Rightarrow ~\forall ~ b\in H,~\exists~c\in H\text{ s.t. }bc=e~(c=b^{-1})                                                        \\
		               & \therefore ~\forall~ a,b\in H,m(a,b^{-1})=ab^{-1}\in H\text{ due to closure under } m                                                    \\
		~                                                                                                                                                         \\
		"\Leftarrow":  & \text{ As a subset of } G,\text{ associativity implied.}                                                                                 \\
		               & ~ H \text{ non-empty } \therefore ~\exists~h\in H.\text{ Second condition implies } aa^{-1}\in H\Rightarrow e\in H.\text{ (unit exists)} \\
		               & \text{ w.l.o.g. we can impose that the subset }H\text{ is not the trivial group, ie. }H\neq(\lbrace h=e\rbrace,m),                       \\
		               & ~\Rightarrow ~\exists~h\in H\text{ s.t. } h\neq e.\Rightarrow eh^{-1}=h^{-1}\in H.\text{ (inverses exist)}                               \\
		               & ~\text{for any 2 elements } a,b\in H, ab^{-1}\in H\text{ and } b^{-1}\Rightarrow a,b^{-1}\in H\text{ so } ab\in H. \text{ (closure)}     \\
		               & ~\therefore H\leq G.
	\end{align*}
\end{proof}
\hrule
\vspace{2mm}
\begin{example}~
	\begin{enumerate}
		\item Every group $G$ has a trivial subgroup, $(\lbrace e\rbrace,m)\leq G$, and an improper subgroup, $G\leq G$.
		\item $n\mathbb{Z}=\lbrace nk|k\in\mathbb{Z}\rbrace\subseteq \mathbb{Z}$ is a subgroup of $(\mathbb{Z},+)$.
		\item Given $x\in G$, $\langle x\rangle=\lbrace x^n|n\in \mathbb{Z}\rbrace\subseteq G$ is the subgroup of $G$ generated by $x$.
		\item Let $G\times X\rightarrow X$ be a group action, and $s\in X$. The \textbf{stabilizer of $s$}, $G_s:=\lbrace g\in G|g\cdot s=s\rbrace$, forms a subgroup of $G$.
		\item Let $\varphi:G\rightarrow H$ be a group homomorphism. Then
		      \begin{itemize}
			      \item $\mathrm{ker }~\varphi:=\lbrace g\in G | \varphi(g)=e\rbrace$
			      \item $\mathrm{im }~\varphi:=\varphi(G)=\lbrace \varphi(g)\in H | g\in G\rbrace$
		      \end{itemize}
		      are both subgroups, of $G$ and $H$ respectively.
	\end{enumerate}
\end{example}
\newpage
\section{Lecture 8: Cyclic Groups and Subgroups.}
\subsection{Cyclic Groups}
\begin{definition}
	A group $H$ is \textbf{cyclic} if it can be generated by a single element. That is to say $\exists~ x\in H$ s.t. $H=\langle x\rangle=\lbrace x^n|n\in \mathbb{Z}\rbrace$.\\
\end{definition}
\begin{theorem}
	If  $H=\langle x\rangle$, then $|H|=|x|$  \footnote{The order of element $x\in G$ can also be defined as the order of the subgroup which it generates.}. Also
	\begin{enumerate}
		\item if $|H|=n<\infty$, then the elements $e,x,x^2,...,x^{n-1}$ are all distinct and $H=\lbrace e,x,...,x^{n-1}\rbrace$.
		\item if $|H|=\infty$ then $H=\lbrace ...,x^{-2},x^{-1},e,x,x^2,...\rbrace$ and $x^a\neq x^b$ if $a\neq b$. \\
	\end{enumerate}
\end{theorem}
\begin{proof}
	\begin{align*}
		|x|=n:~      &                                                                                                             \\
		             & \text{We first show }H=\langle x \rangle\subseteq\lbrace e,x,...,x^{n-1}\rbrace.                            \\
		             & \text{ Let } x^m\in H,~m=kn+r, ~0\leq r\leq n-1                                                             \\
		             & x^m=x^{kn+r}=x^n\cdot x^n \cdot ... \cdot x^n \cdot x^r = e\cdot x^r=x^r                                    \\
		             & \Rightarrow ~ H=\langle x\rangle=\lbrace x^r|0\leq r\leq n-1\rbrace=\lbrace e,x,...,x^{n-1}\rbrace.         \\
		             & ~                                                                                                           \\
		             & \text{To show } x^i\neq x^j~\forall~i,j\text{ s.t. }0\leq i,j\leq n-1,                                      \\
		             & \text{ assume } x^i=x^j,~\Rightarrow x^{i-j}=e. \text{ But, }0\leq i-j\leq n-1,\text{ contradicting } |x|=n \\
		             & ~                                                                                                           \\
		|x|=\infty:~ &                                                                                                             \\
		             & x^i\neq x^j~\forall~ i\neq j\text{ shown as in the first case.}
	\end{align*}
\end{proof}
\begin{theorem}
	Let $H=\langle x\rangle$, then
	\begin{itemize}
		\item if $|H|=n$, there exists a group isomorphism defined by
		      \begin{align}
			      \varphi:~ & \mathbb{Z}/n\mathbb{Z}\rightarrow H \\
			                & k\mapsto x^k
		      \end{align}
		\item and if $|H|=\infty$, there exists group isomorphism defined by
		      \begin{align*}
			      \varphi:~ & \mathbb{Z}\rightarrow H \\
			                & k\mapsto x^k
		      \end{align*}
	\end{itemize}
\end{theorem}
\begin{proof}~
	\begin{itemize}
		\item $\varphi $ well defined:\\
		      Finite order: $k\equiv l (\mathrm{mod}~n)\Rightarrow k=l+mn,~m\in\mathbb{Z}.$\\
		      Then $x^k=x^{l+mn}=x^l(x^n)^m=x^l$. No equivalent terms in $\mathbb{Z}$ for non-finite order.
		\item $\varphi$ group homomorphism:\\
		      $\varphi(k_1+k_2)=x^{k_1+k_2}=x^{k_1}\cdot x^{k_2}=\varphi(k_1)\cdot\varphi(k_2)$
		\item  $\varphi$ bijective:\\
		      Follows from previous proposition. Also provable through existence of left, right inverses.
	\end{itemize}
\end{proof}
\newpage
\subsection{Subgroups of Cyclic Groups}
\begin{theorem}
	Let $G=\langle x\rangle$ be cyclic and $H\leq G$ be a subgroup. Then $H$ is also cyclic.\\
\end{theorem}
\begin{proof}
	Trivial if $H=\lbrace e\rbrace$. Assume not the case, and let $l=\mathrm{min}\lbrace m\in\mathbb{Z}^{>0}|x^m\in H\rbrace $. Then it is obvious that $\langle x^l\rangle\subseteq H$ (closure of subgroup). We then let $x^k \in H$ be an arbitrary element of $H$. We can write $k=l\cdot k^{\prime}+r,~0\leq r\leq l-1$, then $x^r=x^{k-lk^{\prime}}=x^k(x^l)^{k^{\prime}}\in H$ (closure of $H$). But $r<l$, and $l$ is the minimum greater than $0$. Hence, $r=0,x^r=e$, and $x^k=(x^l)^{k^{\prime}}\in \langle x^l\rangle$.
	Thus, $H\subseteq\langle x^l\rangle$, but $\langle x^l\rangle\subseteq H$, and so $H=\langle x^l\rangle$.
\end{proof}
\begin{theorem}
	Let $G=\langle x\rangle$ be an infinite cyclic group ($|G|=\infty$). Then the assignment $n\mapsto\langle x^n\rangle$ defines a bijection from $\mathbb{N}$ and subgroups of G.\\
\end{theorem}
\begin{proof}
	By Prop. 8.2.1, each subgroup of $G$ is of the form $\langle x^n\rangle,~ n\in\mathbb{Z}$. Since $\langle x^{-n}\rangle = \langle x^n\rangle$, we can assume $n\in \mathbb{N}$ (hence, map is surjective). Suppose $\langle x^n\rangle=\langle x^m\rangle$, then
	\begin{align*}
		             & x^n=x^{km} \\
		\Rightarrow~ & n=km
	\end{align*}
	Similarly $m=k^{\prime}n$. Then $n=kk^{\prime}n$, but $k,k^{\prime}\in \mathbb{N}$, so $k=k^{\prime}=1$ and $m=n$ (map is then injective). Then since the outlined map is both injective and surjective, it is bijective.
\end{proof}
\hrule
\vspace{2mm}
\begin{remark}
	From Thm. 8.1.1, $G=\langle x\rangle$ s.t. $|G|=\infty$ is isomorphic to $\mathbb{Z}$. Since each element $x^n$ is mapped to $n$, each subgroup $\langle x^n\rangle$ is mapped to the corresponding subgroup generated by $n$, $\lbrace ..., -n-n, -n, 0, n, n+n,...\rbrace$ which is written as $n\mathbb{Z}=\lbrace n\cdot k | k\in \mathbb{Z}\rbrace$.\\
\end{remark}
\hrule
\newpage
\section{Lecture 9: Euclidean Algorithm}
\begin{definition}
	We say $m$ is a divisor of $n$ if $\exists ~k\in\mathbb{Z}$ s.t. $n=km$ and write $m|n$ in that case.\\
\end{definition}
\hrule
\vspace{2mm}
\begin{example}~
	\begin{itemize}
		\item $1|n~\forall n\in \mathbb{Z}$
		\item $d|m,~d|n ~\Rightarrow ~d|m\pm n$
		\item $n|0,~\forall n$
		\item $d|n ~\Rightarrow ~|d|\leq |n|,~if n\neq 0$
		\item $n|n,~\forall n$
	\end{itemize}
\end{example}

\hrule
\vspace{2mm}
\begin{definition}
	For $m,n\in\mathbb{Z}$ we define the greatest common divisor, $\mathrm{gcd}(m,n):=(m,n):=\mathrm{max}\lbrace d\in\mathbb{Z}^{>0}|~d|m,~d|n\rbrace$, we set $(0,0)=0$.
\end{definition}
\begin{lemma}~
	\begin{enumerate}
		\item $(m,n)=(n,m)$
		\item $(m,n)=(m,n+am)~\forall ~a\in \mathbb{Z}$
		\item $(m,n)=(r,n),~\forall ~r\equiv m ~(\mathrm{mod}~n)$
		\item $(m,0)=|m|$
	\end{enumerate}
\end{lemma}
\begin{proof}
	\begin{enumerate}
		\item Trivial, from definition of g.c.d., changing order does not change set of common denominators.
		\item It is sufficient for this proof to show that the sets $A=\lbrace d\in\mathbb{Z}^{>0}|~d|m,~d|n\rbrace$ and $B=\lbrace d\in\mathbb{Z}^{>0}~|~d|m,~d|n+am\rbrace$, for any $n\in\mathbb{Z}$, are equal.\\

		      For any $d\in A,~d|m,~d|n$. We can then write $m=dp,~n=dq$. It follows that $m+an=d(p+aq)$, and so $d|m+an$. Thus, $A\subseteq B$.\\
		      For any $d\in B$, an analogous argument shows that $B\subseteq A$. Therefore $A=B$, and so $\mathrm{max}\lbrace d\in\mathbb{Z}^{>0}|~d|m,~d|n\rbrace=\mathrm{max}\lbrace d\in\mathbb{Z}^{>0}|~d|m,~d|n+am\rbrace$, and so $(m,n)=(m,n+am)~\forall~n\in\mathbb{Z}$.
		\item Since $r\equiv m~(\mathrm{mod}~n)$, $(r,n)=(m+an,n)=(m,n)$ by the 2nd part of Lemma 9.1.
		\item For $d\in\mathbb{Z}$ s.t. $d|m$,it must be that $d|0$ also (0 divisible by all integers). Therefore, $\lbrace d\in\mathbb{Z}~|~d|m\rbrace \subseteq \lbrace d\in\mathbb{Z}~|~d|0\rbrace$ and so $(m,0)=\mathrm{max}\lbrace d\in\mathbb{Z}~|~d|m,~d|0\rbrace=\mathrm{max}\lbrace d\in\mathbb{Z}~|~d|m\rbrace=|m|$.
	\end{enumerate}
\end{proof}
\begin{theorem}[Euclidean Algorithm]
	\label{EuclidAlg}
	Let $m,n\in\mathbb{Z}$, then $\exists~a,b$ s.t
	\begin{equation}
		(m,n)=am+bn
	\end{equation}
\end{theorem}
\hrule
\vspace{2mm}
The procedure of Euclid's algorithm to find $(m,n)$ for $m,n\in\mathbb{Z}$, where we can presume $m>n$ w.l.o.g. (swapping), involves subtracting the maximum multiples of $n$ from $m$ s.t the remainder is non negative . This makes use of the second part of Lemma 9.1. The procedure is repeated until one of the remainders is zero, in which case the g.c.d. is $(m,n)=(r_{n-1},0)=|r_{n-1}|$. $r_{n-1}|$ is the remainder produced after $n-1$ iterations of the procedure.\\
\hrule
\begin{proof}
	For $m,n\in\mathbb{Z}$, to find $(m,n)$, we use Lemma 9.1. First we can assume $m\geq n$, then $(m,n)=(n,r_1)$ where $r_1\equiv m~\mathrm{mod}~n$. Since $0\leq r_1\leq |n|$, we can replace $n$ with $r_2\equiv n~\mathrm{mod}~|r_1|$. This is repeated until $r_{l+1}=0=r_{l-1}~\mathrm{mod}~r_l$. Then
	\begin{align*}
		(m,n)=(r_l,r_{l+1})=(r_l,0)=|r_l|
	\end{align*}
	Since $r_{i+1}+k_{i+1}r_i=r_{i-1}$,
	\begin{align*}
		\begin{pmatrix}
			r_i \\
			r_{i+1}
		\end{pmatrix}
		=
		 & \underbrace{\begin{pmatrix}
				               0 & 1       \\
				               1 & k_{i+1}
			               \end{pmatrix}}
		\begin{pmatrix}
			r_{i-1} \\
			r_i
		\end{pmatrix}                \\
		 & ~~~~~~A_i
	\end{align*}
	Thus
	\begin{align*}
		\begin{pmatrix}
			r_l \\0
		\end{pmatrix}
		 & =A_l\cdot A_{l-1}\cdot...\cdot A_1\cdot A_0\cdot\begin{pmatrix}
			                                                   m \\n
		                                                   \end{pmatrix} \\
		 & =\begin{pmatrix}
			    a & b \\
			    c & d
		    \end{pmatrix}
		\cdot \begin{pmatrix}
			      m \\n
		      \end{pmatrix}
	\end{align*}
\end{proof}
\newpage
\section{Lecture 10: Subgroups of Finite Cyclic Groups}
\begin{theorem}
	\label{subgoffinitecycG1}
	Let $G=\langle x\rangle$ be a cyclic group of order $|G|=n$. Then
	\begin{enumerate}
		\item $\langle x^l\rangle =\langle x^{(l,n)}\rangle$
		\item $|\langle x^l\rangle|=\frac{n}{(l,n)}$
	\end{enumerate}
	In particular there is a one-to-one correspondence
	\begin{align*}
		A=\lbrace d\in\mathbb{Z}^{>0}~|~d|n\rbrace ~ & \leftrightarrow ~B=\lbrace H \subseteq G | H\leq G\rbrace \\
		d\in A                                       & \mapsto\langle x^d\rangle
	\end{align*}
\end{theorem}
\begin{proof}~
	\begin{enumerate}
		\item It is required to show that the two groups are subsets of each other and so are equivalent.
		      \begin{align*}
			      ``\subseteq" ~ & d=(l,n)\Rightarrow~\exists k\in\mathbb{Z}^{>0}\text{ s.t. } l=kd. \\
			                     & x^l=x^{kd}=(x^d)^k\in\langle x^d\rangle.                          \\
			                     & \Rightarrow~\langle x^l\rangle\subseteq \langle x^d\rangle.       \\
			      ~              & ~                                                                 \\
			      ``\supseteq" ~ & \text{By proposition }	\ref{EuclidAlg},~d=al+bn.                  \\
			                     & x^d=(x^l)^a\cdot(x^n)^b=(x^l)^a\in\langle x^l\rangle              \\
			                     & \Rightarrow \langle x^d\rangle \subseteq \langle x^l\rangle       \\
			      ~              & ~                                                                 \\
			      \Rightarrow~   & \langle x^d\rangle = \langle x^l\rangle
		      \end{align*}
		\item For some $\langle x^l\rangle$, we can assume $l=d\cdot k$, where $d=(l,n)$, by 1. The order of the group generated by $\langle x^d\rangle$ is the smallest $k\in \mathbb{N}$ s.t. $(x^d)^k=e$ or $dk=mn$, for $m\in\mathbb{Z}$. Since $d|n$, we can let $m=1$ w.l.o.g., and so $k=\frac{n}{d}=\frac{n}{(l,n)}$.
	\end{enumerate}
	Define
	\begin{align*}
		\varphi: \lbrace d\in\mathbb{Z}^{>0}~|~d|n\rbrace & \rightarrow \lbrace H \subseteq G | H\leq G\rbrace \\
		d\mapsto\langle x^d\rangle
	\end{align*}
	Each subgroup of G is cyclic, ie. $\forall~H\leq G,~H=\langle x^l\rangle,~l\in\mathbb{Z}$. By proposition \ref{subgoffinitecycG1}, 1., $\langle x^l\rangle=\langle x^{(l,n)}\rangle$. Since $(l,n)|n,~(l,n)\in \lbrace d\in\mathbb{Z}^{>0}~|~d|n\rbrace$ and so $\varphi$ is surjective.\\
	\indent Suppose $\varphi(d_1)=\varphi(d_2)$, then by proposition \ref{subgoffinitecycG1}, 2., $\frac{n}{d_1}=\frac{n}{d_2}$. Therefore $d_1=d_2$. Thus, $\varphi$ is injective.\\
\end{proof}
\hrule
\vspace{2mm}
\begin{remark}
	It is also shown that group $G$ has a unique subgroup of order $\frac{|G|}{d}$ for each pos. divisor $d$ of $|G|$.\\
\end{remark}
\begin{corollary}
	For $x\in G,~\langle x^k\rangle =\langle x\rangle$, then $(k,|x|)=1$.\\
\end{corollary}
\begin{proof}
	\begin{align*}
		|\langle x^k\rangle| & =|\langle x\rangle|=|x|                                       \\
		                     & =\frac{|x|}{(k,|x|)}~\text{ by Prop. \ref{subgoffinitecycG1}}
	\end{align*}
	It follows that $(k,|x|)=1$.\\
\end{proof}
\begin{definition}
	Let $G$ be a group. We define the \textbf{group of automorphisms} of $G$ to be
	\begin{equation}
		\mathrm{Aut}(G)=\lbrace \varphi:G\rightarrow G~|~\varphi \text{ a group isomorphism}\rbrace
	\end{equation}
\end{definition}
\begin{lemma}
	$\mathrm{Aut}(G)\leq S_G$, in particular, $\mathrm{Aut}(G)$ is a group under composition.\\
\end{lemma}
\begin{proof}
	The symmetric group is defined in Eq. \ref{eq: SymmetricGroup}. The symmetric group on $G$ is thereby the group of bijective maps from $G$ to itself. This does not require the maps to be homomorphisms, which is the requirement which the group of automorphisms of $G$ places on these maps.\\
	\indent Let $\varphi\in\mathrm{Aut}(G)$, then $\varphi$ is a group isomorphism, $\varphi:G\rightarrow G$. Then $\varphi$ is bijective, and thus, $\varphi\in S_G$. Hence, $\mathrm{Aut}(G)\subseteq S_G$.\\
	\indent It is then required to show $\mathrm{Aut}(G)$ is a group under composition. The unit of $S_G$ is $\mathrm{id}$, since $\mathrm{id}_G(g_1) \mathrm{id}_G(g_2)=g_1g_2$ and $\mathrm{id}_G$ is bijective, $\mathrm{id}_G$ is a group isomorphism. Thus $\mathrm{id}_G\in\mathrm{Aut}(G)$.\\
	\indent Since each $\varphi\in\mathrm{Aut}(G)$ is bijective we can assume the existence of $\varphi\in S_G$ (bijective). Then since $\varphi$, $\varphi(g_1)\varphi(g_2)=\varphi(g_1g_2)~\forall~g_1,g_2\in G$ we can say $\varphi^{-1}(\varphi(g_1)\varphi(g_2))=g_1g_2=\varphi^{-1}(\varphi(g_1))\varphi^{-1}(\varphi(g_2))$. Since $\varphi(g)\in G$, it follows that $\varphi^{-1}$ is a group isomorphism, and so $\varphi\in\mathrm{Aut}(G)$.\\
	\indent For $\varphi_1,\varphi_2\in\mathrm{Aut}(G)$, $\varphi_1(g_1g_2)=\varphi_1(g_1)\varphi_1(g_2)$ and so $\varphi_2(\varphi_1(g_1g_2))=\varphi_2(\varphi_1(g_1))\varphi_2(\varphi_1(g_2))$, then $\varphi_2\circ\varphi_1\in\mathrm{Aut}(G)$.\\
	\indent Therefore $\mathrm{Aut}(G)$ is a group under composition.\\
\end{proof}
\begin{theorem}
	The group $\mathrm{Aut}(\mathbb{Z}/n\mathbb{Z})$ is isomorphic to the multiplicative group of integers modulo n.
	\begin{equation}
		\mathrm{Aut}(\mathbb{Z}/n\mathbb{Z})\cong\mathbb{Z}/n\mathbb{Z}^{\times}:=\lbrace k\in\mathbb{Z}/n\mathbb{Z}~|~(k,n)=1\rbrace
	\end{equation}
	Where $\mathbb{Z}/n\mathbb{Z}^{\times}$ has group multiplication defined by $\overline{k_1}\cdot \overline{k_2}=\overline{k_1\cdot k_2}$, and unit $\overline{1}$.\\
\end{theorem}
\begin{proof}
	Define map
	\begin{align*}
		\Psi:~\mathrm{Aut}(\mathbb{Z}/n\mathbb{Z}) & \rightarrow\mathbb{Z}/n\mathbb{Z}^{\times} \\
		\varphi                                    & \mapsto\varphi(1)
	\end{align*}
	\begin{itemize}
		\item $\Psi$ is well defined:
		      \begin{align*}
			      \langle 1 \rangle = \mathbb{Z}/n\mathbb{Z} & \Rightarrow \langle \varphi(1)\rangle = \mathbb{Z}/n\mathbb{Z}=\langle 1 \rangle \\
			                                                 & \Rightarrow (n,\varphi(1))=1 \text{, by Cor. 10.1.1}                             \\
			                                                 & \Rightarrow \varphi(1)\in\mathbb{Z}/n\mathbb{Z}^{\times}
		      \end{align*}
		      The first step here relies on the fact that $n=|1| =|\varphi(1)|$ which can be shown from the requirements of a automorphism. It is then easy to show that each automorphism is mapped to an element of $\mathbb{Z}/n\mathbb{Z}^{\times}$.\\
		\item $\Psi$ is injective:
		      \begin{align*}
			      bzd
		      \end{align*}
		\item $\Psi$ is surjective:
		      \begin{align*}
			      bzd
		      \end{align*}
		\item $\Psi$ is a group homomorphism:
		      \begin{align*}
			      \Psi(\varphi_1\circ\varphi_2) & =\varphi_1\circ\varphi_2(1) =\varphi_1(\varphi_2(1))=\
	\end{itemize}
\end{proof}





\end{document}